\documentclass[11pt,stdletter,sigleft]{newlfm}
\usepackage{charter}

%\widowpenalty=1000
%\clubpenalty=1000

\newlfmP{dateskipafter=1pt}
\newlfmP{headermarginskip=1pt}
\newlfmP{sigsize=5pt}
\newlfmP{dateskipafter=20pt}
\newlfmP{addrfromphone}
\newlfmP{addrfromemail}
\PhrPhone{Phone}
\PhrEmail{Email}

\namefrom{Jordan R. Willis Ph.D.}

\addrfrom{9022 NW 86th Terrace \\ Kansas City, MO 64153}
\phonefrom{816-674-5340}
\emailfrom{jwillis0720@gmail.com}

\addrto{Max Delbr\"{u}ck Professor of Biology and HHMI Investigator Pamela Bj\"{o}rkman \\ California Institute of Technology 1200 E. California Blvd. \\ Pasadena, CA 91125 }

\greetto{Dear Professor Bj\"{o}rkman,}
\closeline{Thank you for your consideration,}

\begin{document}
\begin{newlfm}
I am a recently completed Doctoral student at Vanderbilt University Medical Center in the research group of James Crowe and Jens Meiler. My thesis was focused on computer aided antibody design using the software suite \textsc{rosetta}. I worked at the interface of the wet- and dry-labs in which I carried my designs over to the wet-lab to characterize. To that end I gained proficiency in a wide range of computational techniques including molecular modeling, data mining,  computer programming as well as experimental techniques including molecular cloning, expression, viral neutralization assays and biophysical characterization. The antibodies I designed were primarily against HIV and Influenza.


As HIV researcher and antibody designer, your work is of great interest to me. Particularly, the recent collaborations with the Nussenzweig group. The rational design of NIH45-46\textsuperscript{G54W} to mimic CD4 I found fascinating as I was also focusing on the rational redesign of published broadly neutralizing antibodies to increase breadth and specificity. I used \textsc{rosettadesign} to increase the stability of PG9 in complex with it's native epitope thereby increasing potency and breadth. I'm also highly interested in your work with germline recognition of HIV. Much of my thesis focused on the HIV na\"{i}ve repertoire and it's ability to recognize and neutralize V1/V2 epitopes. I used a combination of bioinformatics, high-throughput sequencing, and molecular modeling to identify long HCDR3 loops in the repertoire that can bind and neutralize HIV thereby mimicking the functionality of PG9.

I feel my expertise at the interface between computational antibody design, bioinformatics, HIV immunology and experimental characterization would be an invaluable asset to the  Bj\"{o}rkman group.  If you would like, I would be happy to visit your lab in person, or to talk on the phone, I would also be happy to arrange for letters of reference to be sent. Please let me know if you have any questions or would like to schedule a meeting.


\end{newlfm}
\end{document}