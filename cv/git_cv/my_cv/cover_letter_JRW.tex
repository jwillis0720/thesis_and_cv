% \documentclass[11pt,stdletter,sigleft]{newlfm}
% \usepackage{charter}

% %\widowpenalty=1000
% %\clubpenalty=1000

% \newlfmP{dateskipafter=1pt}
% \newlfmP{headermarginskip=1pt}
% \newlfmP{sigsize=5pt}
% \newlfmP{dateskipafter=20pt}
% \newlfmP{addrfromphone}
% \newlfmP{addrfromemail}
% \PhrPhone{Phone}
% \PhrEmail{Email}

% \namefrom{Jordan R. Willis Ph.D.}

% \addrfrom{9022 NW 86th Terrace \\ Kansas City, MO 64153}
% \phonefrom{816-674-5340}
% \emailfrom{jwillis0720@gmail.com}

% \addrto{Max Delbr\"{u}ck Professor of Biology and HHMI Investigator Pamela Bj\"{o}rkman \\ California Institute of Technology 1200 E. California Blvd. \\ Pasadena, CA 91125 }

% \greetto{Dear Professor Bj\"{o}rkman,}
% \closeline{Thank you for your consideration,}

% \begin{document}
% \begin{newlfm}
% I am a recently completed Doctoral student at Vanderbilt University Medical Center in the research group of James Crowe and Jens Meiler. My thesis was focused on computer aided antibody design using the software suite \textsc{rosetta}. I worked at the interface of the wet- and dry-labs in which I carried my designs over to the wet-lab to characterize. To that end I gained proficiency in a wide range of computational techniques including molecular modeling, data mining,  computer programming as well as experimental techniques including molecular cloning, expression, viral neutralization assays and biophysical characterization. The antibodies I designed were primarily against HIV and Influenza.


% As HIV researcher and antibody designer, your work is of great interest to me. Particularly, the recent collaborations with the Nussenzweig group. The rational design of NIH45-46\textsuperscript{G54W} to mimic CD4 I found fascinating as I was also focusing on the rational redesign of published broadly neutralizing antibodies to increase breadth and specificity. I used \textsc{rosettadesign} to increase the stability of PG9 in complex with it's native epitope thereby increasing potency and breadth. I'm also highly interested in your work with germline recognition of HIV. Much of my thesis focused on the HIV na\"{i}ve repertoire and it's ability to recognize and neutralize V1/V2 epitopes. I used a combination of bioinformatics, high-throughput sequencing, and molecular modeling to identify long HCDR3 loops in the repertoire that can bind and neutralize HIV thereby mimicking the functionality of PG9.

% I feel my expertise at the interface between computational antibody design, bioinformatics, HIV immunology and experimental characterization would be an invaluable asset to the  Bj\"{o}rkman group.  If you would like, I would be happy to visit your lab in person, or to talk on the phone, I would also be happy to arrange for letters of reference to be sent. Please let me know if you have any questions or would like to schedule a meeting.


% \end{newlfm}
% \end{document}




% %Nussensweig
% \documentclass[11pt,stdletter,sigleft]{newlfm}
% \usepackage{charter}

% \widowpenalty=1000
% \clubpenalty=1000

% \newlfmP{dateskipafter=.1pt}
% \newlfmP{dateskipbefore=.1pt}
% \newlfmP{headermarginskip=.1pt}
% \newlfmP{sigsize=.1pt}
% \newlfmP{dateskipafter=.1pt}
% %\newlfmP{addrfromphone}
% %\newlfmP{addrfromemail}
% %\PhrPhone{Phone}
% %\PhrEmail{Email}

% \namefrom{Jordan R. Willis, Ph.D.}

% %\addrfrom{9022 NW 86th Terrace \\ Kansas City, MO 64153}
% %\phonefrom{816-674-5340}
% %\emailfrom{jwillis0720@gmail.com}

% \addrto{Zanvil A. Cohn and Ralph M. Steinman Professor and HHMI Investigator \\ Michel C. Nussenzweig, M.D., Ph.D.\\ 1230 York Ave \\ New York, NY 10065 }

% \greetto{Dear Professor Nussenzweig,}
% \closeline{Thank you for your consideration,}

% \begin{document}
% \begin{newlfm}
% I am a recently completed Doctoral student at Vanderbilt University Medical Center in the research group of James Crowe and Jens Meiler. My thesis was focused on computer aided antibody design using the software suite \textsc{rosetta}. I worked at the interface of the wet- and dry-labs in which I carried my designs over to the wet-lab to characterize. To that end I gained proficiency in a wide range of computational techniques including molecular modeling, data mining,  computer programming as well as experimental techniques including molecular cloning, expression, viral neutralization assays and biophysical characterization. The antibodies I designed were primarily against HIV and Influenza.


% As a member of the Crowe laboratory, we are well read in your work. I'm interested in many aspects of your research. The rational design of antibodies was a focus of my thesis, so the high-impact work from your NIH45-46\textsuperscript{G54W} and additional NIH45-46\textsuperscript{G54W} resistant variants are of interest to me. I worked on PG9 redesigned variants that enhance breadth and specificity by stabilizing the complex. These mutations were recovered through computer aided design rather than CD4 mimicry, but closely resembles your aims to enhance specificity. I'm also highly interested in your work on germline antibodies in the context of HIV. Much of my thesis was focused on how germline sequences adopt a large structural repertoire as well as minimal mutations needed for HIV-na\"{i}ve long HCDR3 loops to mimic V1/V2 targeting antibodies. Your work was pivotal in my motivation to pursue this hypothesis as you showed somatic mutations of framework region of many of the broadly neutralizing antibodies were necessary to neutralize HIV. Further, I would like to pursue more \textit{in vivo} characterization of my antibody designs, and your pioneering work on humanization of mice is of great interest.


% I feel my expertise at the interface between computational antibody design, bioinformatics, HIV immunology and experimental characterization would be an invaluable asset to the Nuzzenzweig group.  If you would like, I would be happy to visit your lab in person, or to talk on the phone, I would also be happy to arrange for letters of reference to be sent. Please let me know if you have any questions or would like to schedule a meeting.


% \end{newlfm}
% \end{document}



%Lanzavechia
\documentclass[11pt,stdletter,sigleft]{newlfm}
\usepackage{charter}

\widowpenalty=1000
\clubpenalty=1000

\newlfmP{dateskipafter=.1pt}
\newlfmP{dateskipbefore=.1pt}
\newlfmP{headermarginskip=.1pt}
\newlfmP{sigsize=.1pt}
\newlfmP{dateskipafter=.1pt}
%\newlfmP{addrfromphone}
%\newlfmP{addrfromemail}
%\PhrPhone{Phone}
%\PhrEmail{Email}

%\namefrom{Jordan R. Willis, Ph.D.}

%\addrfrom{9022 NW 86th Terrace \\ Kansas City, MO 64153}
%\phonefrom{816-674-5340}
%\emailfrom{jwillis0720@gmail.com}

\addrto{Professor Human Immunology, ETH Zurich \\ Director Institute for Research in Biomedicine \\ Antonio Lanzavecchia, M.D.\\ Via Vincenzo Vela 6 \\ CH-6500 Bellinzona, Switzerland }

\greetto{Dear Professor Lanzavecchia,}
%\closeline{Thank you for your consideration,}

\begin{document}
\begin{newlfm}
I am a recently completed Doctoral student at Vanderbilt University Medical Center in the research group of James Crowe and Jens Meiler. My thesis was focused on computer aided antibody design using the software suite \textsc{rosetta}. I worked at the interface of the wet- and dry-labs in which I carried my designs over to the wet-lab to characterize. To that end I gained proficiency in a wide range of computational techniques including molecular modeling, data mining,  computer programming as well as experimental techniques including molecular cloning, expression, viral neutralization assays and biophysical characterization. The antibodies I designed were primarily against HIV and Influenza.


As a member of the Crowe laboratory, and an antibody designer, I'm well read in your work. The rational design of antibodies was a focus of my thesis, I'm highly interested in polyspecific and highly potent antibodies from rationale engineering. One aim of my research was developing a ``multi-state design'' algorithm for the programming suite \textsc{rosetta}. For this, you can design for multiple epitopes at once without compromising potency to any of the epitopes of interest. This is very similar to what your team in collaboration with Luca Varani accomplished by using rationale engineering to tune the specificity of antibodies to various DenV serotypes. This resembles my own work in which I fine tuned specificity of PG9 using \textsc{rosettadesign}. In contrast to your work, I used \textsc{rosetta} to select mutations that seemed beneficial to multiple HIV variants. I feel this type of computational design is exactly what your group needs in validating and engineering antibodies for cross-reactivity. You have shown that you are an expert on this subject such is your work with cross-pathogen HRSV and HMPV binder, and expanded breadth in Influenza group A and B. Please see my CV for additional details that echo much of the work you have done including high-throughput bioinformatic analysis.


I feel my expertise at the interface between computational antibody design, bioinformatics, HIV immunology and experimental characterization would be an invaluable asset to the Lanzavecchia group.  If you would like, I would be happy to visit your lab in person, or to talk on the phone, I would also be happy to arrange for letters of reference to be sent. Please let me know if you have any questions or would like to schedule a meeting.
\\
\\ \\
Thank you for your consideration,\\
Jordan R. Willis, Ph.D.
\end{newlfm}
\end{document}
