\section{Research Experience}
\cventry{2014-}{Research Associate}{The Scripps Research Institute}{La Jolla}{CA}{My research focuses on fusing computation modeling, high-throughput sequencing, and library display technologies in order to design immunogen to drive a protetive response against HIV. These immunogens work by priming very rare B cells that have antibodies which contain a long anionic HCDR3. I have used bioinformatic analysis to recognize patterns in long HCDR3s that may be potentially targetbale by a tailored immunogen.}
\cventry{2009-2014}{Graduate Research Assistant}{Vanderbilt University Medical Center}{Nashville}{TN}{My thesis involves using computational design to answer specific questions in viral HIV immunology, with a focus on antibody design. The computational work was accomplished in the Meiler laboratory while the experimental laboratory work was conducted in the Crowe laboratory. My thesis work can be divided into four parts:%
\begin{enumerate}[itemsep=1mm,topsep=1mm,leftmargin=.25in,parsep=1mm]
\item Multi-state antibody design to interrogate mechanisms for antibody polyspecificity. How do antibodies use a limited sequence repertoire to bind many antigens?
\item Molecular mechanisms of CD4-binding site escape for HIV-1 gp120. How does gp120 escape neutralization by two CD4-binding site-specific, broadly neutralizing antibodies, VRC01 and b12? We used computational characterization, homology modeling, and biophysical characterization to test our hypothesis.
\item Determine how closely antibody sequences from HIV-na\"{i}ve individuals are to broadly neutralizing antibodies against HIV. Using computational modeling, high-throughput sequencing, and bioinformatics tools, I designed antibodies from HIV-na\"{i}ve donor sequence pools that mimic broadly neutralizing antibodies with exceptionally long HCDR3s.
\item Computational design of antibodies with increased neutralization breadth against diverse natural variants of the influenza hemagglutinin stem.
\end{enumerate}}
\vspace{1mm}
\cventry{2007-2008}{Undergraduate Research Fellow}{University of Missouri, Department of Chemistry}{Columbia}{MO}{Lead optimization drug discovery of hypoxic-cell targeting molecules that treat tumors. Using the pharmacaphore Tirapazamine as a scaffold, I used combinatorial synthesis techniques to add organic groups and evaluate structural activity relationships.} 
%Teaching Experience
\section{Teaching Experience}
\cventry{2012}{Instructor - Rosetta teaching worksop}{Vanderbilt University}{Nashville}{TN}{Developed protocol, taught background and gave hands-on demonstration for design for Rosetta teaching workshop 2012}
\cventry{2011}{Instructor - Rosetta teaching worksop}{Vanderbilt University}{Nashville}{TN}{Developed protocol, taught background and gave hands-on demonstration for protein docking for Rosetta teaching workshop 2011}
\cventry{2007-2008}{Laboratory teaching assistant}{Northwest Missouri State University}{Maryville}{MO}{General Chemistry 1 and 2}
\cventry{2005-2007}{Tutor - Talent Development Center}{Northwest Missouri State University}{Maryville}{MO}{Tutored in the following subjects: Physics I and II, general chemistry, organic chemistry, analytical chemistry, physical chemistry, statistics, algebra, and calculus.}
