\chapter*{Summary}
\addcontentsline{toc}{chapter}{Summary}
\vspace{7mm}
This document is the culmination of my work on antibody design. Primarily, my target system is HIV with some work on Influenza. It is divided up into orthogonal publications, with each publication having incorporated an aspect of antibody design. Here I use the modeling suite \rosetta~whenever I mention \silico~solutions to design problems. All computational work was done by me as well as validation with the experimental characterization.

The introduction in chapter \ref{chap:chapter1} outlines four pieces of background knowledge that must be considered for the remaining chapters to become clear. I first detail antibodies in general, as this document only considers immunoglobulins as the design target of interest. I detail their purpose and how they are diversified to create an immunoglobulin repertoire. I next detailed the pandemic of HIV, its structure, the current state of an efficacious vaccine, and describe the broadly neutralizing antibodies that have been characterized to date. I also describe the molecular modeling suite \rosetta~and briefly cover the \rosetta~energy function. I detail a particular application in \rosetta~known as \rosettadesign, which is the focus of my thesis. I then describe how \rosettadesign~has been used in translational medicine. Lastly I detail the current state of the field, the difficulties in molecular modeling, the challenges in protein design, and how this work can be used to aid in these challenges. Briefly, I describe how antibody design is used to answer questions about basic science to implications for vaccine strategies. It is here that I tie antibody design with all aspects of my research.

The beginning of my research starts in chapter \ref{chap:chapter2}. This part of my research aims to answer a basic question in immunology. Here I used \rosettadesign~to investigate the molecular basis for polyspecificity. It is known that a finite set of antibodies is able to accommodate a nearly limitless antigen space. Using design, I investigated which sequences are ideal to bind multiple antigens or single antigens in a protocol I used called multi-state design.

Using the strategies and principles built upon in chapter \ref{chap:chapter2}, chapter \ref{chap:chapter3} focuses on a novel vaccine paradigm. Here I used antibody design to interrogate the HIV-\naive~donor antibody repertoire with a simple goal in mind: Does the HIV-\naive~antibody repertoire contain antibodies that resemble known broadly neutralizing antibodies? This paradigm focuses on a structural mimicry rather than a sequence identity, which not only allows me to use \rosettadesign~and homology modeling as a tool to investigate this hypothesis, but mandates it, showing the necessity of molecular modeling. All antibodies found in this chapter were made experimentally and characterized to corroborate computational findings.

In chapter \ref{chap:chapter4}, I used \rosettadesign~to show that the antibody PG9, which is known to be one of the most broad and potent antibodies against HIV-1 characterized to date, still has room for improvement. Mutations were returned from \rosettadesign~which were predicted to enhance breadth and specificity. These mutations were tested experimentally and did indeed corroborate our hypothesis that even the most broad and potent antibodies could be improved with careful computational design and analysis.

Finally, chapter \ref{chap:chapter5} details future directions I foresee for these project. These strategies are generalizable and can be applied to any given antibody/antigen system and may even extend to any given protein-protein interface. In addition, I consider reasons for design failures, imperfect sequence recovery, and antibodies that failed to corroborate \silico~predictions. I give an idea of many future experiments that could be used to take of advantage of the information detailed in this document. I also detail  some of my other work on viral escape assessed by computational modeling and broadly neutralizing mAbs to Influenza which were in various stages of completion.