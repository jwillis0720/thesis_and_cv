\chapter*{Summary}
\addcontentsline{toc}{chapter}{Summary}
\vspace{7mm}

The work presented in this thesis is primarily composed of previously published manuscripts, or drafts of manuscripts being prepared for submission.

Chapter \ref{chap:introduction} is an introductory chapter outlining the background and significance of the research described in the dissertation.
Part of this chapter, specifically section \ref{sec:rosetta_intro}, was partially based on text originally published as "Practically useful: what the Rosetta protein modeling suite can do for you" \citep{Kaufmann:2010ea}, To which I was a contributing co-first author.
The remainder of chapter \ref{chap:introduction} is original to this dissertation.

Chapter \ref{chap:nv_kbp} was originally published as "Design of Native-like Proteins through an Exposure-Dependent Environment Potential" \citep{DeLuca:2011gg}.
This chapter describes the implementation, optimization, and benchmarking of a novel energy term for protein design.
This energy term is a \ac{KBP} which computes a score based on the propensity of an amino acid existing at a given degree of burial within a protein.
Additionally, a new metric for assessing the quality of designed proteins based on their \ac{PSSM} score was introduced.
The weights of the RosettaDesign energy function were then optimized using Particle Swarm Optimization.
The resulting optimized energy function incorporating the new \ac{KBP} showed significantly improved protein designs using the metrics of protein sequence recovery, sequence composition bias, and the new \ac{PSSM} recovery metric.

Chapter \ref{chap:lowres_paper} is a draft of a manuscript currently being prepared for publication on which I will be the sole first author.
This manuscript describes improvements in the scientific and speed performance of RosettaLigand.
Specifically, the initial placement step of the RosettaLigand algorithm was rewritten to more efficiently sample the ligand binding site space using a Metropolis Monte Carlo algorithm that simultaneously translates and rotates the protein.
Additionally, the code used to translate and rotate the ligand within the binding site was carefully optimized to improve computational efficiency.
These changes resulted in a reduction in the number of models needed to reliably produce a high quality binding pose from 1000 to 150, a decrease in the amount of time needed to produce a model from 45-90 seconds to 5-15 seconds, and an increase of 10-15\% in the number of protein targets to which ligands can be successfully docked.
The combination of reduced run time and improved scientific performance makes it possible for RosettaLigand to be used to perform \ac{vHTS} for the first time.

Chapter \ref{chap:rosetta_hts} is a draft of a manuscript currently being prepared for publication on which I will be the sole first author.
This manuscript describes a \ac{vHTS} protocol which uses predictions generated by the RosettaLigand protocol described in chapter \ref{chap:lowres_paper} as input into an \ac{ANN} trained to predict whether ligands have active binding affinity.
This chapter describes the design of a set of data for training the \ac{ANN} model which is both diverse in protein and chemical space and also balanced in chemical space between active and inactive compounds.
Additionally, a set of \ac{RDF} based fingerprint descriptors representing the geometric and chemical properties of the protein-ligand interface are introduced.
The goal of this work is to eventually produce a model which is capable of accurately classifying compounds based on activity.
The \ac{ANN} models described in this chapter substantially improve upon the performance of RosettaLigand alone when making classification predictions within the cross-validation data set.
However, these methods are less able to make predictions outside of the cross-validation data set.  When applied to a large benchmark, they are generally less likely to have significantly worse than random classification performance compared to the use of RosettaLigand interface scores alone as a classifier.
The maximum classification performance obtained is also lower than with RosettaLigand alone, indicating that while the method shows promise as a means of improving the reliability of ligand docking methods, additional research is required to refined the method.

Chapter \ref{chap:conclusion} outlines the findings and future directions of the research presented in the prior chapters.

Chapter \ref{chap:hts_preprocess} is an appendix describing the design and usage of a software processing pipeline I developed to rapidly parameterize large numbers of ligands for input into RosettaLigand.
This pipeline makes it possible to prepare hundreds of thousands of ligands in a matter of hours in a semiautomated fashion, and is a critical part of the work performed in \ref{chap:lowres_paper} and \ref{chap:rosetta_hts}.

Chapter \ref{chap:sql_appendix} is an appendix describing the design of a software system for storing and retrieving protein structural information generated by Rosetta using a \ac{SQL} database.
This system was developed in collaboration between myself and three other members of the RosettaCommons, and decreases the disk space requirements of RosettaLigand by approximately 99\% in a \ac{vHTS} protocol.

Chapter \ref{chap:lowres_appendix} is an appendix describing the development of novel scoring grids for use with the RosettaLigand initial placement algorithm.
These scoring grids did not significantly improve upon the performance of RosettaLigand, and the development of effective \ac{KBP} based scoring functions for initial placement remains an active area of research. 

Chapter \ref{chap:nv_capture_appendix} is a protocol capture document describing the method by which the experiments in chapter \ref{chap:nv_kbp} can be reproduced.
This document was originally published as a supplement to the manuscript \citep{DeLuca:2011gg}.

Chapter \ref{chap:lowres_capture} is a protocol capture document describing the method by which the experiments in chapter \ref{chap:lowres_paper} can be reproduced.
It will be published along side the manuscript as supplemental information.

Chapter \ref{chap:hts_appendix} is a protocol capture document describing the method by which the experiments in chapter \ref{chap:rosetta_hts} can be reproduced.
It will be published along side the manuscript as supplemental information.