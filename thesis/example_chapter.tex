\chapter{Introduction}
	This document is intended to provide a template for writing a master's thesis or dissertation. In particular, this document correctly adheres to all of Vanderbilt University's formatting standards for the Electrical Engineering and Computer Science Department as of December 2010. If you are using this template to submit a thesis to a different institution or department, please review your department's submission guidelines as this template may not be applicable.
	
	In order to see the LaTeX code used to generate this document, please refer to the VanderbiltDissertationFormat.tex file located in this directory. 
	
\section{Figures, Tables, Equations, and Algorithms}
	In addition to fulfilling the necessary formatting requirements, this document will also show examples of how insert images, tables, algorithms, and references into your LaTeX documentation. Each will be demonstrated in its own subsection to illustrate subsection use.

\subsection{Inserting Figures}
Figure insertion can vary depending on what you need to insert. The code associated with Figure \ref{fig:texlogo} illustrates how to insert single figures, which is the method you will likely use the most often. 

\begin{figure}
\centering
\includegraphics[width=2in]{latex_logo.png}
\caption{A sample figure.}
\label{fig:texlogo}
\end{figure}

More complex figure insertion (e.g. inserting multiple figures that have a single caption) can also be achieved. Figure \ref{fig:testedShapes} illustrates how to insert multiple figures.

\begin{figure}[htp]
  \begin{center}
    \subfloat[Square]{\label{fig:label1}\includegraphics[width = 1in]{square}}
    \subfloat[Hexagon]{\label{fig:hex}\includegraphics[width = 1in]{hexagon}}
    \subfloat[Triangle]{\label{fig:tri}\includegraphics[width = 1in]{triangle}}\\
    \subfloat[Star]{\label{fig:star}\includegraphics[width = 1in]{star}}
    \subfloat[House]{\label{fig:house}\includegraphics[width = 1in]{house}}
    \subfloat[Square Path]{\label{fig:sqpath}\includegraphics[width = 1in]{squarePath}}\\
 \end{center}
  \caption{Multiple figure insertion example}
  \label{fig:testedShapes}
\end{figure}

\subsection{Using Tables}
Tables can be entered directly into your LaTeX document; however, inserting tables directly into your dissertation .tex file can result in unwanted clutter. It is generally cleaner to write the table in its own .tex file and then insert the .tex file into your dissertation .tex file. This is the method used by this template. An example table can be seen in Table \ref{table:test}.

\begin{table}
\scriptsize
\renewcommand{\tabcolsep}{0.09cm}
\centering
\input{tableTest2.tex}
\caption{A sample table.}
\label{table:test}
\end{table}

\subsection{Equations and Equation Arrays}
A simple equation can be seen in Equation \ref{eqn:slope}:
\begin{equation}
\centering
m = \frac{y_{2} - y_{1}}{x_{2} - x_{1}}.
\label{eqn:slope}
\end{equation}

An equation array can be seen in Equation \ref{eqn:generalBez}: 

\begin{eqnarray}
\lefteqn{B(t) =  \sum_{i=0}^n \binom{n}{i} (1-t)^{n-i}t^{i}\mathbf{P}_i }\nonumber \\ 
& = & (1-t)^{n}\mathbf{P}_0 + \binom{n}{1}t\mathbf{P}_1+ \cdots  \nonumber \\
 &\cdots & + \binom{n}{n-1}(1-t)^{n-1}\mathbf{P}_{n-1} + t^{n}\mathbf{P}_n, \phantom{ttt} t \in [0,1]
\label{eqn:generalBez}
\end{eqnarray}

\subsection{Algorithms}
An example of a Bayes' Filter \citep*{Thrun2005} algorithm can be seen in Algorithm \ref{alg:bayes}. Note that there are numerous methods for inserting algorithms. This template just demonstrates one.

\begin{algorithm}{$\mathbf{Bayes Filter}$}[$bel(x_{t-1})$, $u_t$, $z_t$]
\caption{The Bayes filter algorithm.}
\label{alg:bayes}
\begin{algorithmic}[1]
\FOR{all $x_t$}
\STATE $\overline{bel}(x_t) = \int p(x_t | u_t, x_{t-1}) bel(x_{t-1})dx_{t-1}$
\STATE $bel(x_t) = \eta p(z_t | x_t) \overline{bel}(x_t)$
\ENDFOR
\RETURN $bel(x_t)$
\end{algorithmic}
\end{algorithm}
