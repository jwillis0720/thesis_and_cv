%In addition you need a substantial (10-20 pages) introductory chapter detailing
%  Significance (Why was the research done needed?) and
%  Innovation (Why was the research done novel? How does it relate to competing methods?) of the body of work in your thesis.
%This chapter could ideally come form a review article you have written.
\chapter{Introduction}

% history and state of the ligand docking field
\section{The history of ligand docking}
Attempts to model and predict protein-drug interactions date began shortly after the publication of the x-ray structure of hemoglobin, with Beddell et. al. Publishing a proof of concept method for structure based drug discovery in 1976\citep{BEDDELL:1976go}.
The method developed by Beddell relied on the manual placement of physical molecular models into a scale model of the hemoglobin electron density, which allowed the authors to identify novel compounds with millimolar activity. 
While the identified compounds were relatively poor by modern standards, and the method of manual placement into physical models did not provide a means of postulating mechanism of action, the authors recognized the value of the new technique, saying:
\begin{quote}
It has been common practice to design new drugs by modifying the chemical structure of a known substance which has the desired biological properties, and this procedure has imposed severe restraints on the choice.
However, it is not necessary for the novel compounds to be related to the original substance when the structure of the receptor site is already known. 
\end{quote}
It is remarkable that this observation on the state of rational drug discovery continues to be relevant, nearly 40 years after it was originally made. 

After the advent of relatively inexpensive general purpose computers in the early 1980s, the promise of accurate and rapid computational design of novel small molecules has driven a wide array of research into improved methods for predicting protein-ligand interfaces.
A successful protein-ligand docking tool must solve two basic problems: sampling and scoring.
To effectively solve the sampling problem, the software must be able to efficiently explore both the rigid space of the protein binding site, as well as the conformational space of both 	the protein and the ligand.
To effectively solve the scoring problem, a score function must be developed which can rapidly distinguish between energy favorable and unfavorable conformations.
Solving both of these problems has proven highly challenging, although great progress has been made. 

In 1982, Kuntz et al. published DOCK, one of the earliest computational tools for modeling protein-ligand interactions\citep{Kuntz:1982wx}.  Dock used a relatively simple energy function which modeled repulsive forces as hard spheres, and a rough approximation of hydrogen bonding which favored binding positions in which hydrogen bond donor groups on the ligand were within 3-5\AA\ of acceptor nitrogens and oxygens on the protein backbone.
In concept, the DOCK algorithm is similar to the manual placement method described by by Beddell et al above. 
The program uses the van der waals radii of the protein and ligand atoms to create "space filled" representations of both the receptor pocket and the ligand.
Pairs of protein and ligand spheres are then considered systematically, and the set of pairings which minimizes sphere overlap is selected.
This algorithm is driven almost entirely by shape complementarity, and effectively models the "lock and key" hypothesis of protein-ligand binding, in which a rigid protein is matched with a rigid ligand.

In 1984, Goodford et al published GRID, a computational method for predicting energetically favorable protein-ligand binding conformations\citep{Goodford:1985bf}.
GRID differed from previous attempts structure based drug discovery in that it used chemical information rather than relying entirely on receptor fit. 
Specifically, it assessed the protein-ligand interaction using an empirical energy function consisting of a Lennard-Jones term, electrostatic term, and hydrogen bonding term.
This energy function was precomputed as a 3 dimensional grid overlaid on the ligand binding site.
Thus, the total score of the ligand could be rapidly assessed as the sum of the grid squares the atoms are located in.
Pre-computation of the scoring grid enabled many ligand conformations and compositions to be rapidly assessed, and the addition of chemical information in addition to shape proved more effective than simply evaluating shape complementarity.

While the GRID method proved reasonably effective, it had a number of shortcomings which limited it's effectiveness.
The physics based force field used was relatively rudimentary, and the limited set of chemical probes used to create the grid.
Additionally, accurate docking into a full-atom grid based model requires a high degree of precision in the position of the protein atoms, which limits the effectiveness of such a model in cases where the accuracy of the protein structure is lower.

In the years following the publication of DOCK and GRID, additional experimental study of protein structure began to suggest that the rigid body lock and key model was not adequate for the modeling of protein-ligand interactions.
It had long been suspected\citep{KOSHLAND:1958wa} that enzymes and receptors may be flexible to accommodate the fit of small molecules (the so called "induced fit" hypothesis), however in 1995, Nicklaus et al.\citep{Nicklaus:1995tu} published work suggesting that small molecules also undergo substantial conformational shift on binding.
This conclusion was arrived at by comparing the geometry of flexible small molecules observed bound to proteins with the geometry of the same small molecules when crystallized in the absence of a protein, or when computationally minimized using molecular mechanics.
The results of this study indicated that while the conformations of rigid structures typically differed by < 0.1 \AA\ RMSD between the bound an unbound context, flexible ligands typically differed significantly, frequently by several angstroms.
Furthermore, the difference in RMSD between bound and unbound ligands was strongly correlated with the number of rotatable bonds in the ligand, with an R$^{2}$ correlation of 0.82.
In response to this research, the development of newer protein-ligand docking methods began to focus on the flexibility of the system.
While flexibility had previously been avoided due to the inherent increase in complexity associated with modeling it, it became apparent that it was a critical component of protein-ligand interaction.

FlexX\citep{Rarey:1996hf} and Genetic Optimization of Ligand Docking (GOLD)\citep{Jones:1997bl}, are two of the early methods which attempted to model ligand flexibility.
FlexX represents the ligand binding site using a set of interaction sites, which are defined as surfaces surrounding hydrogen bond donors and acceptors, metals and metal acceptors, aromatic rings, methyls and amides.
An empirical scoring function is used to score ligand conformations based on the distance and angle between defined protein and ligand interaction sites.
FlexX uses an incremental construction algorithm to model ligand flexibility.
An initial core fragment of the ligand is placed in the binding site using an incremental construction algorithm, and the additional fragments necessary to complete the ligand are then placed such that they can connect to the initial fragment and minimize the energy function score. 
GOLD, on the other hand, relies on the user providing a reasonable initial position for the ligand inside the protein binding site.
From that initial position, a genetic algorithm\citep{Jones:1995vw} was used to optimize the rotation angles of both the ligand and the interacting protein side-chains. 
The genetic algorithm makes it possible to rapidly find a high quality local minimum without the exhaustive sampling of bond angles that had made the problem previously intractable.
As a result of this new sampling technique, GOLD was able to successfully recover the correct binding conformation in 71 out of 100 X-Ray crystal structures in a benchmarking study. 

In 2004, Glide\citep{Friesner:2004hm} was published as a novel method for protein-ligand docking aimed at the screening of large libraries of small molecules.
To improve the speed of the algorithm, Glide models the receptor site using a set of cartesian scoring grids, and keeps the receptor atoms fixed.
This allows the ligand to be rapidly scored, making it possible for a large number of ligand positions to be evaluated.
Glide performs a set of exhaustive searches along at cartesian grid overlaid on the receptor binding site.
To reduce the amount of sampling required, the step size of the grid is reduced over the course of the search process, beginning with a 2.0 \AA\ pitch grid.
Additionally, a set of filters based on the ChemScore\citep{Eldridge:1997tm} energy function are used to progressively filter the set of allowable binding orientations by increasingly detailed metrics.
After an initial starting position is accepted, The conformational space of the ligand is exhaustively searched, and the final pose is energy minimized.
The use of a grid representation for the energy function makes it possible to to screen large numbers of compounds very rapidly, making Glide a popular choice for virtual screening studies\citep{Yilmaz:2013dj,Bauer:2013de}.

% history of RosettaLigand
\section{The history of RosettaLigand}
RosettaLigand was originally published in 2006\citep{Meiler:2006vj} as a protein-ligand docking tool based off of the previously published RosettaDock\citep{Gray:2003uk} protein-protein docking tool.
The original RosettaLigand docking algorithm took advantage of the knowledge based energy function used by RosettaDock.
The use of a knowledge based potential rather than a physics based potential is advantageous as knowledge based potentials are capable of indirectly modeling effects that are difficult to model directly. %TODO:cite this
Additionally, RosettaLigands ability to rapidly optimize protein side-chain geometry\citep{Barth:2007cw} made it possible to model protein-ligand interactions with full atomic detail.
While RosettaLigand was frequently able to accurately predict the binding orientation ligands\citep{Meiler:2006vj}, it was unable to model backbone or ligand flexibility, which have long been suspected to be critical for protein-ligand binding\citep{Yang:2014dm,KOSHLAND:1958wa}.
To rectify this situation, further extensions were made to RosettaLigand by Davis et al\citep{Davis:2009bf} which allowed RosettaLigand to fully consider the flexibility of all parts of both the protein and the ligand.
A blind benchmarking study comparing the pose recovery performance of the 2009 version of RosettaLigand suggested that overall it performed similarly to other major ligand docking tools\citep{Davis:2009fx}.
A notable conclusion of this study is that while most of the tools studied have a similar performance overall, the performance in predicting docking pose for individual protein targets varies wildly.
This inconstant performance between protein targets and protein docking tools is seem in other studies as well. 

One of the hypothetical advantages of a knowledge based energy function is the ability to accurately model complex physical effects without a direct physical model.
In principle, this, combined with the ability to model both backbone and sidechain flexibility would make RosettaLigand well suited to the docking of ligands into comparative models or other low resolution protein structures. 
To assess this, a benchmarking study was performed in which small molecules with known binding positions were docked into homology models generated in the CASP experiment\citep{Kaufmann:2012ck}.
The results of this benchmark demonstrated that in most of the tested cases, Rosetta was able to generate low energy binding positions within 2.0\AA\ of the crystallographic binding site.

In addition to benchmarking studies, Rosetta has been used to develop models of ligand binding in GPCRs.
A comparative model of hSERT was created based on the dSERT crystal structure. 
S- and R-citalopram were docked into this comparative model using RosettaLigand, and the resulting predicted binding poses were used to design mutational studies to identify residues critical for S-citalopram binding.
Rosetta was able to correctly predict that Y95 and E444 formed protein-ligand interactions critical to binding\citep{Combs:2011db}.  
Similarly, RosettaLigand was used to model the binding of Positive Allosteric Modulators in a comparative model of mGlu$_{5}$\citep{Turlington:2013et}.
In this case, the predictions made by Rosetta were used to guide mutation and radioligand binding studies, the results of which were used to further refine models.
These models made it possible to map out critical interactions between Positive Allosteric Modulators and the mGlu$_{5}$ binding site even in the absence of crystal structure information.


\section{Computational Ligand Docking has inconsistent predictive power} 

A common thread running through the research described above is the difficulty of docking ligands into some proteins.
For every protein-ligand method developed, some percentage of protein-ligand interfaces cannot be effectively predicted.
While the predictions generated by protein-ligand docking has made some major scientific contributions to drug discovery and molecular modeling, the unreliability of the method has historically constrained its usefulness.

In 2006, a diverse set of 81 protein targets, each with a diverse set of known active and predicted inactive ligands was assembled as the DEKOIS 2.0 dataset\citep{Bauer:2013de}.
Glide, GOLD and Autodock Vina were used to screen this dataset, and the pROC AUC enrichment for each target and each screening method was computed.
The results of this benchmark showed a wide range in the predictive ability of the three screening methods.  
While all 3 docking methods had strong predictive power against some protein targets (COX2, KIF11), there were several cases in which no method had predictive power (HSP90, QPCT), and more cases in which some methods were able to make accurate predictions while others were not (COX1, ROCK-1).
Furthermore, it was not possible for the authors to identify straightforward patterns to predict which protein targets could be successfully screened against and which could not. 
The phenomenon of structure based vHTS methods having inconsistent performance depending on the protein target has been replicated in other studies.
For example, the Directory of Useful Decoys, Enhanced (DUD-E) benchmark set was screened using DOCK, and the resulting predictions exhibited similar inconsistencies to those seen in the DEKOIS 2.0 study\citep{Mysinger:2012hu}.

\section{The causes of ligand docking inconsistencies are unclear}

% existing ligand based techniques are limited in scope
\section{The use of machine learning methods has been valuable for ligand based vHTS}

As a result of the inconsistent and unpredictable performance of structure based vHTS methods described in the previous section, ligand based vHTS is generally more common.
A wide range of methods exist for generating QSAR models based on a set of known active and inactive compounds.


% ligand based techniques are hard to use if you have a novel target

% Integration of ligand and structure based methods

% Review of use of Neural networks for vHTSm