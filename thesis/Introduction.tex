%In addition you need a substantial (10-20 pages) introductory chapter detailing
%  Significance (Why was the research done needed?) and
%  Innovation (Why was the research done novel? How does it relate to competing methods?) of the body of work in your thesis.
%This chapter could ideally come form a review article you have written.
\chapter{Introduction}

% history and state of the ligand docking field
\section{The history of ligand docking}
Attempts to model and predict protein-drug interactions date began shortly after the publication of the x-ray structure of hemoglobin, with Beddell et. al. Publishing a proof of concept method for structure based drug discovery in 1976\citep{BEDDELL:1976go}.
The method developed by Beddell relied on the manual placement of physical molecular models into a scale model of the hemoglobin electron density, which allowed the authors to identify novel compounds with millimolar activity. 
While the identified compounds were relatively poor by modern standards, and the method of manual placement into physical models did not provide a means of postulating mechanism of action, the authors recognized the value of the new technique, saying:
\begin{quote}
It has been common practice to design new drugs by modifying the chemical structure of a known substance which has the desired biological properties, and this procedure has imposed severe restraints on the choice.
However, it is not necessary for the novel compounds to be related to the original substance when the structure of the receptor site is already known. 
\end{quote}
It is remarkable that this observation on the state of rational drug discovery continues to be relevant, nearly 40 years after it was originally made. 

Over the intervening decades, rational structure based drug discovery has remained a tremendously challenging problem, though great progress has been made.
The advent of relatively inexpensive general purpose computers in the early 1980s made computational molecular modeling possible, with PJ Goodford pioneering the development of early methods for computational protein-ligand docking.
In 1984, Goodford et al published GRID, a computational method for predicting energetically favorable protein-ligand binding conformations\citep{Goodford:1985bf}.
GRID differed from previous attempts structure based drug discovery in that it used chemical information rather than relying entirely on receptor fit. 
Specifically, it assessed the protein-ligand interaction using an empirical energy function consisting of a Lennard-Jones term, electrostatic term, and hydrogen bonding term.
This energy function was precomputed as a 3 dimensional grid overlaid on the ligand binding site.
Thus, the total score of the ligand could be rapidly assessed as the sum of the grid squares the atoms are located in.
Pre-computation of the scoring grid enabled many ligand conformations and compositions to be rapidly assessed, and the addition of chemical information in addition to shape proved valuable. 

The promise of accurate and rapid computational design of novel small molecules has driven a wide array of research into improved methods for predicting protein-ligand interfaces. 
In the time following 
%TODO: continue with brief review of modern major methods, as well as results

In 2006, a diverse set of 81 protein targets, each with a diverse set of known active and predicted inactive ligands was assembled as the DEKOIS 2.0 dataset\citep{Bauer:2013de}.
Glide, GOLD and Autodock Vina were used to screen this dataset, and the pROC AUC enrichment for each target and each screening method was computed.
The results of this benchmark showed a wide range in the predictive ability of the three screening methods.  
The diversity of the benchmarking results provide a few useful insights about the current state of computational ligand docking. 
%TODO: continues this, what is going on here

% history of RosettaLigand
\section{The history of RosettaLigand}
RosettaLigand was originally published in 2006\citep{Meiler:2006vj} as a protein-ligand docking tool based off of the previously published RosettaDock\citep{Gray:2003uk} protein-protein docking tool.
The original RosettaLigand docking algorithm took advantage of the knowledge based energy function used by RosettaDock.
The use of a knowledge based potential rather than a physics based potential is advantageous as knowledge based potentials are capable of indirectly modeling effects that are difficult to model directly. %TODO:cite this
Additionally, RosettaLigands ability to rapidly optimize protein side-chain geometry\citep{Barth:2007cw} made it possible to model protein-ligand interactions with full atomic detail.
While RosettaLigand was frequently able to accurately predict the binding orientation ligands\citep{Meiler:2006vj}, it was unable to model backbone or ligand flexibility, which have long been suspected to be critical for protein-ligand binding\citep{Yang:2014dm,KOSHLAND:1958wa}.
To rectify this situation, further extensions were made to RosettaLigand by Davis et al\citep{Davis:2009bf} which allowed RosettaLigand to fully consider the flexibility of all parts of both the protein and the ligand.
A blind benchmarking study comparing the pose recovery performance of the 2009 version of RosettaLigand suggested that overall it performed similarly to other major ligand docking tools\citep{Davis:2009fx}.
A notable conclusion of this study is that while most of the tools studied have a similar performance overall, the performance in predicting docking pose for individual protein targets varies wildly.
This inconstant performance between protein targets and protein docking tools is seem in other studies as well. 

One of the hypothetical advantages of a knowledge based energy function is the ability to accurately model complex physical effects without a direct physical model.
In principle, this, combined with the ability to model both backbone and sidechain flexibility would make RosettaLigand well suited to the docking of ligands into comparative models or other low resolution protein structures. 
To assess this, a benchmarking study was performed in which small molecules with known binding positions were docked into homology models generated in the CASP experiment\citep{Kaufmann:2012ck}.
The results of this benchmark demonstrated that in most of the tested cases, Rosetta was able to generate low energy binding positions within 2.0\AA\ of the crystallographic binding site.

In addition to benchmarking studies, Rosetta has been used to develop models of ligand binding in GPCRs.
A comparative model of hSERT was created based on the dSERT crystal structure. 
S- and R-citalopram were docked into this comparative model using RosettaLigand, and the resulting predicted binding poses were used to design mutational studies to identify residues critical for S-citalopram binding.
Rosetta was able to correctly predict that Y95 and E444 formed protein-ligand interactions critical to binding\citep{Combs:2011db}.  
Similarly, RosettaLigand was used to model the binding of Positive Allosteric Modulators in a comparative model of mGlu$_{5}$\citep{Turlington:2013et}.
In this case, the predictions made by Rosetta were used to guide mutation and radioligand binding studies, the results of which were used to further refine models.
These models made it possible to map out critical interactions between Positive Allosteric Modulators and the mGlu$_{5}$ binding site even in the absence of crystal structure information.

\section{}

\section{The history of structure based virtual High Throughput Screening (vHTS)} 

The development of inexpensive high performance computing has made the screening of large compound libraries with protein-ligand docking tools possible in recent years.
Structure based vHTS promises to enable researchers to identify novel inhibitors f

% existing ligand based techniques are limited in scope

% ligand based techniques are hard to use if you have a novel target

% Integration of ligand and structure based methods

% Review of use of Neural networks for vHTSm